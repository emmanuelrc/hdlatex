% !TeX spellcheck = en_US
\documentclass[%
 12pt,           % Schriftgroesse
 english,        % wird an andere Pakete weitergereicht
 a4paper,        % Seitengroesse
 DIV14,          % Textbereichsgroesse (siehe Koma Skript Dokumentation !)
 twoside,        % Für zweiseitigen Druck
 thesis,         % thesis, summary, normal
 impberklaerung, % Wechsel zwischen zwei Erklärungen
]{hdprotokoll}
% -------------------------------------------------------------------------


\usepackage[main=english, ngerman]{babel} % Spracheinstellung
\usepackage[T1]{fontenc}                  % T1 Schrift Encoding
\usepackage[utf8]{inputenc}               % Font Encoding, benoetigt fuer Umlaute

\usepackage{geometry}
\usepackage{textgreek}  % Grichisch im Text
\usepackage{upgreek}    % gerade Griechische Buchstaben
\usepackage{textcomp}   % Zusatzliche Symbole (Text Companion font extension)
\usepackage{amsmath}    % Mathesymbole
\usepackage{lmodern}
\usepackage{booktabs}   % für Tabellen
\usepackage{color}
\usepackage[table]{xcolor}
\usepackage{graphicx}
\usepackage{grffile}
\usepackage{xspace}
\usepackage[detect-all=true]{siunitx} % Si units
% microtype for better type setting
\usepackage[activate={true,nocompatibility},final,tracking=true,kerning=true,spacing=true,factor=1100,stretch=10,shrink=10]{microtype}   
\usepackage{hyperref}


% Unterfigures und die Beschriftung dieser
\usepackage{subfig}
\renewcommand*\thesubfigure{\arabic{subfigure}} 


%%%%%%%%%%%%%%%%%%%%%%%%%%%%%%%%%%%%%%%%%%%%%%%%%%%%%%%%%%%%%%%%%%
% Fonts Time New Roman similar font:
% wenn du Times New roman haben willst kommentiere die nächsten
% zwei zeilen aus. Sie ersetz die schöne LaTeX CMU Serif mit
% Times New Roman und CMU sans Serif mit Helvetica
%\usepackage{mathptmx}
%\usepackage{helvet}
%%%%%%%%%%%%%%%%%%%%%%%%%%%%%%%%%%%%%%%%%%%%%%%%%%%%%%%%%%%%%%%%%%


%%%%%%%%%%%%%%%%%%%%%%%%%%%%%%%%%%%%%%%%%%%%%%%%%%%%%%%%%%%%%%%%%%
% bibtex
\usepackage[backend=biber,natbib=true,style=authoryear,maxcitenames=2,maxbibnames=200,firstinits=true, uniquename=false,doi=false,isbn=false,url=false]{biblatex}
\addbibresource{bibliothek.bib}
\DefineBibliographyStrings{ngerman}{
 andothers = {{et\,al\adddot}}, 
}
%%%%%%%%%%%%%%%%%%%%%%%%%%%%%%%%%%%%%%%%%%%%%%%%%%%%%%%%%%%%%%%%%%

%%%%%%%%%%%%%%%%%%%%%%%%%%%%%%%%%%%%%%%%%%%%%%%%%%%%%%%%%%%%%%%%%%
% Glossar
\usepackage[nonumberlist,acronym]{glossaries}
\makeglossaries
\input{glossar.g}
%%%%%%%%%%%%%%%%%%%%%%%%%%%%%%%%%%%%%%%%%%%%%%%%%%%%%%%%%%%%%%%%%%


%%%%%%%%%%%%%%%%%%%%%%%%%%%%%%%%%%%%%%%%%%%%%%%%%%%%%%%%%%%%%%%%%%
% Titelblatt
\title{HD Protokoll}
\author{Your name}
\date{dd.mm.yyyy}
\titleheader{Ruprecht-Karls-Universität Heidelberg\\
    Fakultät für Biowissenschaften\\
    Masterstudiengang Molekulare Biotechnologie}
\supervisor{Betreut durch Dr. Who}

% Template für eure wissenschaftlichen Vorträge (summary). 
% Bei dieser vorlage muss die Matrikelnummer nach vorne aus Titelblatt
%\supervisor{\begin{center}\parbox{0cm}{\begin{tabbing}
%                Matrikelnummeraaa as \= 0123456789\kill
%                Name: \> Vorname nachname\\
%                Matrikelnummer: \> 0123456789\\
%                Email: \> name@stud.uni-heidelberg.de
%\end{tabbing}}\end{center}}

\protokolltype{Latex Vorlage}
\institute{free text}
\dateStart{n. August 2017}
\dateEnd{n. August 2017}

%%%%%%%%%%%%%%%%%%%%%%%%%%%%%%%%%%%%%%%%%%%%%%%%%%%%%%%%%%%%%%%%%%
% Masterthesis (Bachelor noch nicht supportet)
\Aname{Erstgutachtername}        % Erstgutachter
\Aabteilung{Abteilung}           % Abteilung
\Ainstitute{dsdsf}               % Institut

\Bname{Erstgutachtername}        % Zweitgutachter
\Babteilung{Abteilung}           % Abteilung
\Binstitute{dsdsf}               % Institut

%%%%%%%%%%%%%%%%%%%%%%%%%%%%%%%%%%%%%%%%%%%%%%%%%%%%%%%%%%%%%%%%%%



\begin{document}
\pagenumbering{roman}            % Die ersten Seitenzahlen sollen römisch sein
\maketitle                       % Titelseite
\cleardoublepage                 % Fang auf einer rechten Seite an

\tableofcontents                 % Inhaltsverzeichnis
%\listofhdtitle % Table of contents for summaries
               % Make a section header like this: 
               % \hdtitle{TITLE}{DATE}{LECTURE/SEMINAR} instead of \section{}

\newpage        			     % Falls erwünscht 
\printglossaries                 % ein Abkürzungsverzeichnis               

\cleardoublepage                 % Fang auf einer rechten Seite an
\pagenumbering{arabic}           % Aber jetzt mit arabischen Seitenzahlen


\section{Section header}
Lorem ipsum dolor sit amet, exerci officiis no mea. Ex pro ubique quidam. Ponderum dissentias sit an, mei an graecis elaboraret. Mei et nisl nihil, iusto copiosae eam no, ad mei melius forensibus reprimique \gls{coli}.

Quod propriae phaedrum vix te, an mei primis verear insolens. Eum case facilisis abhorreant id, eu sea dicant forensibus. Ius ad vocent splendide argumentum. Nihil vivendum vix ut. Per te impedit eleifend, in vis enim laudem ocurreret. Paulo detracto ea sed \citep{hector1999plant}.

Duo agam volutpat in, ne quo suscipit consulatu interpretaris. Ius te unum veri signiferumque. Ancillae deleniti an mei, option praesent explicari in vim. Mel populo vidisse albucius ad, te eam aliquam officiis, sit integre iudicabit scriptorem at. Pri debet scribentur ei, eos at munere verterem moderatius. Vivendo interpretaris usu ex, qui et detracto delectus, his porro prodesset in.

Tollit scaevola at qui, sit soluta nonumes ei. Ad pri iriure scriptorem. No mei ferri oblique dignissim, mei ea oblique signiferumque. Erant atomorum gloriatur an has, exerci animal ne mel, an aliquid intellegebat per. Dicunt iisque labores eu vim.

Quis patrioque eu mea. Mei scripta pertinax at. His at dolorum dignissim. Mel dicit paulo scaevola ei. Sonet eruditi eu mei, ad sed vidit aeterno. His ei quem graece, id audire expetendis est. Maiorum posidonium cum at, nam id pertinacia delicatissimi, noster laoreet cum id.

\printbibliography                % Die Bibliographie



\end{document}